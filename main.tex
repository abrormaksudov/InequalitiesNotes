\documentclass[a4paper,11pt]{article}

\usepackage[top=1in, bottom=1in, left=1in, right=1in]{geometry}
\usepackage{hyperref}
\usepackage{amsmath, amssymb, amsthm}
\usepackage{enumitem}
\usepackage{graphicx}
\usepackage[labelformat=empty]{caption}
\usepackage{tcolorbox}
\tcbuselibrary{breakable}
\usepackage{multicol}
\usepackage{tabularx}
\usepackage{ltablex}
\usepackage{centernot}

\title{Inequalities Notes}
\author{Abror Maksudov}
\date{Last Updated: \today}

\everymath{\displaystyle}

\begin{document}

\maketitle

%
% AM-GM
%
\section{Algebraic Inequalities}
\textbf{Theorem 1 (AM-GM)}. \textit{Let $a_1, \dots, a_n$ be non-negative real numbers. Then:}
\[
\frac{a_1 + \dots + a_n}{2} \geq \sqrt[n]{a_1 \dots a_n}
\]
\textit{with equality if and only if $a_1 = a_2 = \dots = a_n$.} \\[10pt]
%
% Cauchy-Schwarz
%
\textbf{Theorem 2 (Cauchy-Schwarz)}. \textit{Let $a_1, \dots, a_n$, $b_1, \dots, b_n$ be real numbers. Then:}
\[
(a_1^2 + \dots + a_n^2)(b_1^2 + \dots + b_n^2) \geq (a_1b_1 + \dots a_nb_n)^2
\]
%
% Titu's Lemma
%
\textbf{Theorem 3 (Titu's Lemma)}. \textit{Let $a_1, a_2, \dots, a_n$, $b_1, b_2, \dots, b_n$ be positive real numbers. Then:}
\[
\frac{a_1^2}{b_1} + \frac{a_2^2}{b_2} + \dots + \frac{a_n^2}{b_n} \geq \frac{(a_1 + a_2 + \dots + a_n)^2}{b_1 + b_2 + \dots + b_n}  
\]
%
% Young's Inequality
%
\textbf{Theorem 4 (Young's Inequality).} \textit{Let $a, b$ be nonnegative real numbers and if $p, q > 0 $ such that $\frac{1}{p} + \frac{1}{q} = 1$. Then:}
\[
ab \leq \frac{a^p}{p} + \frac{b^q}{q}
\]
\textit{with equality if and only if $a^p = b^q$.} \\[10pt]
%
% Hölder's Inequality
%
\textbf{Theorem 5 (Hölder's Inequality)}. \textit{Let $a_1, \dots, a_n$, $b_1, \dots, b_n$ be positive real numbers. Suppose that  $P > 1$ and $q > 1$ satisfy $\frac{1}{p} + \frac{1}{q} = 1$. Then:}
\[
\left( \sum_{i=1}^n a_i^p \right) ^ \frac{1}{p} \left( \sum_{i=1}^n b_i^q \right) ^ \frac{1}{q} \geq \sum_{i=1}^n a_ib_i
\]
\textit{More generally, let $x_{ij} (i=1, \dots, m, j=1, \dots, n)$ be positive real numbers. Suppose that $w_1, w_2, \dots, w_n$ are positive real numbers satisfying $w_1 + w_2 + \dots + w_n = 1$. Then:}
\[
\prod_{j=1}^{n} \left( \sum_{i=1}^m x_{ij} \right)^{w_j} \geq \sum_{i=1}^m \left( \prod_{j=1}^n x_{ij}^{w_j} \right)
\]
%
% Minkowski Inequality
%
\textbf{Theorem 6 (Minkowski Inequality).} \textit{Let $a_1, \dots, a_n$, $b_1, \dots, b_n$ be positive real numbers. Suppose that $p > 1$. Then:}
\[
\left( \sum_{i=1}^n a_i^p \right)^\frac{1}{p} + \left( \sum_{i=1}^n b_i^p \right)^\frac{1}{p} \geq \left( \sum_{i=1}^n (a_i + b_i)^p \right)^\frac{1}{p}
\]
%
% Generalized Minkowski Inequality
%
\textbf{Theorem 7 (Generalized Minkowski Inequality).} \textit{Let $a_{ij} \geq 0$ for $i=1, \dots, n$ and $j=1, \dots, n$ and let $p>1$. Then:}
\[
\left[ \sum_{i=1}^n \left( \sum_{j=1}^m a_{ij} \right)^p \right]^\frac{1}{p} \leq \sum_{j=1}^m \left( \sum_{i=1}^n a_{ij}^p \right)^\frac{1}{p} 
\]
%
% Chebyshev's Sum Inequality
%
\textbf{Theorem 8 (Chebyshev's Sum Inequality).} \textit{Let $a_1, \dots, a_n$ and $b_1, \dots, b_n$ be real numbers. Then:}
\begin{align*}
    \frac{a_1b_1 + \dots + a_nb_n}{n} &\geq \frac{(a_1 + \dots + a_n)}{n} \frac{(b_1 + \dots + b_n)}{n} \\
    \frac{1}{n} \sum_{i=1}^n a_ib_i &\geq \left( \frac{1}{n} \sum_{i=1}^n a_i \right) \left( \frac{1}{n} \sum_{i=1}^n b_i \right)
\end{align*}
%
% Rearrangement Inequality
%
\textbf{Theorem 9 (Rearrangement Inequality).} \textit{Let $a_1, \dots, a_n$ and $b_1, \dots, b_n$ be real numbers. For any permutation $\sigma$ of $\{1, \dots, n\}$, we have:}
\[
\sum_{i=1}^n a_ib_i \geq \sum_{i=1}^n a_ib_{\sigma(i)} \geq \sum_{i=1}^n a_ib_{n+1-i}
\]
%
% Convex Function
%
\textbf{Definition 1 (Convex Function).} \textit{Suppose that $f$ is a one-variable function defined on $[a, b] \subset \mathbb{R}$. $f$ is called a convex function on $[a, b]$ if and only if for all $x, y \in [a, b]$ and for all $0 \leq t \leq 1$, we have:}
\[
tf(x) + (1 - t)f(y) \geq f(tx + (1-t)y)
\]
%
% Jensen’s Inequality
%
\textbf{Theorem 10 (Jensen’s Inequality).} \textit{Let $f: [a, b] \to \mathbb{R}$ be a convex function. Then for any $x_1, \dots, x_n \in [a, b]$ and non-negative real numbers $w_1, \dots, w_n$ with $w_1 + \dots + w_n = 1$, we have:}
\[
\sum_{i=1}^n w_i f(x_i) \geq f( \sum_{i=1}^n w_i x_i )
\]
%
% Popoviciu's Inequality
%
\textbf{Theorem 11 (Popoviciu's Inequality).} \textit{Let $f: I \to \mathbb{R}$. If $f$ is convex, then for any three points $x, y, z$ in $I$:}
\[
\frac{f(x) + f(y) + f(z)}{3} + f\left(\frac{x + y + z}{3} \right) \geq \frac{2}{3} \left[ f\left( \frac{x + y}{2} \right) + f\left( \frac{y + z}{2} \right) + f\left( \frac{z + x}{2} \right) \right]
\]
%
% Majorization
%
\textbf{Definition 2 (Majorization).} \textit{Given two sequences $(a) = (a_1, a_2, \dots, a_n)$ and  $(b) = (b_1, b_2, \dots, b_n)$ (where $a_i, b_i \in \mathbb{R} \quad \forall i \in \{ 1, 2, \dots, n \}$). We say that the sequence $(a)$ majorizes the sequence $(b)$, and write $(a) \succ (b)$, if the following conditions are fulfilled:}
\begin{align*}
    a_1 \geq a_2 \geq &\dots \geq a_n; \\
    b_1 \geq b_2 \geq &\dots \geq b_n; \\
    a_1 + a_2 + &\dots + a_n = b_1 + b_2 + \dots + b_n; \\
    a_1 + a_2 + &\dots + a_k = b_1 + b_2 + \dots + b_k \quad \forall k \in \{ 1, 2, \dots, n-1 \} \\
\end{align*}
%
% Karamata's Inequality
%
\textbf{Theorem 12 (Karamata's Inequality).} \textit{Let $f: [a, b] \to \mathbb{R}$ be a convex function. Suppose that $(x_1, \dots, x_n) \succ (y_1, \dots, y_n)$ where $x_1, \dots, x_n$, $y_1, \dots, y_n \in [a, b]$. Then:}
\[
\sum_{i=1}^n f(x_i) \geq \sum_{i=1}^n f(y_i)
\]
%
% Weighted AM-GM Inequality
%
\textbf{Theorem 13 (Weighted AM-GM Inequality).} \textit{Let $w_1, \dots, w_n \geq 0$ such that $w_1 + \dots w_n = 1$. For all $x_1, \dots, x_n \geq 0$, we have:}
\[
\sum_{i=1}^n w_i x_i \geq \prod_{i=1}^n x_i^{w_i}
\]
%
% Schur's Inequality
%
\textbf{Theorem 14 (Schur's Inequality).} \textit{Let $x, y, z$ be non-negative real numbers. For any $r > 0$, we have:}
\[
\sum_{cyc} x^r (x - y) (x - z) \geq 0
\]
%
% Generalized Schur's Inequality
%
\textbf{Theorem 15 (Generalized Schur's Inequality).} \textit{Let $a, b, c, x, y, z$ be six non-negative real numbers such that the sequences $(a, b, c)$ and $(x, y, z)$ are similarly sorted. Then:}
\[
x (a - b) (a - c) + y (b - c) (b - a) + z (c - a) (c - b) \geq 0
\]
%
% Newton's Inequality
%
\textbf{Theorem 16 (Newton's Inequality).} \textit{Let $x_1, \dots, x_n$ be non-negative real numbers. Define the symmetric polynomials $s_0, s_1, \dots, s_n$ by $(x + x_1)(x + x_2) \dots (x + x_n) = s_nx^n + \dots + s_1 + s_0$, and define the symmetric averages by $d_i = \frac{s_i}{\binom{n}{i}}$. Then:}
\[
d_i^2 \geq d_{i+1}d_{i-1}
\]
%
% Maclaurin's Inequality
%
\textbf{Theorem 17 (Maclaurin's Inequality).} \textit{Let $x_1, \dots, x_n$ be non-negative real numbers. Define the symmetric polynomials $s_0, s_1, \dots, s_n$ by $(x + x_1)(x + x_2) \dots (x + x_n) = s_nx^n + \dots + s_1 + s_0$, and define the symmetric averages by $d_i = \frac{s_i}{\binom{n}{i}}$. Then:}
\[
d_1 \geq \sqrt[2]{d_2} \geq \sqrt[3]{d_3} \geq \dots \geq \sqrt[n]{d_n}
\]
%
% Muirhead's Inequality
%
\textbf{Theorem 18 (Muirhead's Inequality).} \textit{Suppose that $(a_1, \dots, a_n) \succ (b_1, \dots, b_n)$, and $x_1, \dots, x_n$ are positive real numbers. Then:}
\[
\sum_{sym} x_1^{a_1} x_2^{a_2} \dots x_n^{a_n} \geq \sum_{sym} x_1^{b_1} x_2^{b_2} \dots x_n^{b_n}
\]
\textit{where the symmetric sum is taken over all $n!$ permutations of $(x_1, x_2, \dots, x_n)$.} \\[10pt]
%
% Power Mean Inequality
%
\textbf{Theorem 19 (Power Mean Inequality).} \textit{Let $x_1, \dots, x_n > 0$. The power mean of order $r$ is defined by:}
\[
\left\{
\begin{aligned}
  M_{(x_1, \dots, x_n)}(0) &= \sqrt[n]{x_1 \dots x_n} \\
  M_{(x_1, \dots, x_n)}(r) &= \left( \frac{x_1^r + \dots + x_n^r}{n} \right)^\frac{1}{r}, \quad r \neq 0
\end{aligned}
\right.
\]
\textit{Then, $M_{(x_1, \dots, x_n)} : \mathbb{R} \to \mathbb{R}$ is continuous and monotone increasing.}
%
% Bernoulli’s Inequality
%
\textbf{Theorem (Bernoulli’s Inequality).} \textit{For every real number $r \geq 1$ and real number $x \geq -1$, we have:}
\[
(1 + x)^r \geq 1 + rx
\]
\textit{while for $0 \leq r \leq 1$ and real number $x \geq -1$, we have:}
\[
(1 + x)^r \leq 1 + rx
\]
%
% Aczel's Inequality
%
\textbf{Theorem 20 (Aczel's Inequality).} \textit{Let $a_1, \dots, a_n, b_1, \dots, b_n$ be non-negative real numbers satisfying $a_1^2 \geq a_2^2 + \dots + a_n^2$ and $b_1^2 \geq b_2^2 + \dots + b_n^2$. Then:}
\[
a_1b_1 - (a_2b_2 + \dots + a_nb_n) \geq \sqrt{(a_1^2 - (a_2^2 + \dots + a_n^2))(b_1^2 - (b_2^2 + \dots + b_n^2))}
\]
%
% Huygens Inequality
%
\textbf{Theorem 21 (Huygens Inequality).} \textit{Let $a_1, a_2, \dots, a_n, b_1, b_2, \dots, b_n, w_1, w_2, \dots, w_n$ be positive real numbers such that $w_1 + w_2 + \dots + w_n = 1$. Then:}
\[
\prod_{i=1}^n (a_i + b_i)^{w_i} \geq \prod_{i=1}^n a_i^{w_i} + \prod_{i=1}^n b_i^{w_i}
\]
%
% Heinz Inequality
%
\textbf{Theorem 22 (Heinz Inequality).} \textit{For $a, b > 0$ and $\alpha \in [0, 1], we have:$}
\[
\sqrt{ab} \leq \frac{a^{\alpha}b^{1-\alpha} + a^{1-\alpha}b^{\alpha}}{2} \leq \frac{a + b}{2}
\]

\end{document}
