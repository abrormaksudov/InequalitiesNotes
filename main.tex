\documentclass[a4paper,11pt]{article}

\usepackage[top=1in, bottom=1in, left=1in, right=1in]{geometry}
\usepackage{hyperref}
\usepackage{amsmath, amssymb}
\usepackage{enumitem}
\usepackage{graphicx}
\usepackage[labelformat=empty]{caption}
\usepackage{tcolorbox}
\tcbuselibrary{breakable}
\usepackage{multicol}
\usepackage{tabularx}
\usepackage{ltablex}
\usepackage{centernot}

\title{Inequalities Notes}
\author{Abror Maksudov}
\date{Last Updated: \today}

\everymath{\displaystyle}

\begin{document}

\maketitle
\tableofcontents

\section{Introduction}


\section{Definitions}


\subsection{Majorization}
\begin{tcolorbox}[breakable]
    Let $x = (x_1, x_2, \dots, x_n)$ and $y = (y_1, y_2, \dots, y_n)$ be non-increasing sequences of real numbers. Then $x$ is said to \emph{majorize} $y$, denoted $x \succ y$, if the following conditions are satisfied:
    \begin{enumerate}
        \item $x_1 \geq x_2 \geq \dots \geq x_n$ \emph{and} $y_1 \geq y_2 \geq \dots \geq y_n$\,;
        \item $\sum_{i=1}^n x_i = \sum_{i=1}^n y_i$\,;
        \item $\sum_{i=1}^k x_i \geq \sum_{i=1}^k y_i$ for all $k = 1, 2, \dots, n-1$.
    \end{enumerate}
    Example: $(3,1,0) \succ (2,1,1), \quad (12, 0,0) \succ (4,4,4)$.
\end{tcolorbox}


\subsection{Convex Function}
\begin{tcolorbox}[breakable]
    A function $f : [a,b] \to \mathbb{R}$ is called \textit{convex} (concave up) on $[a,b]$ if and only if for all $x,y \in [a,b]$ and all $\lambda \in [0,1]$, the following inequality holds:
    \[
    \lambda f(x) + (1-\lambda)f(y) \geq f(\lambda x + (1-\lambda)y).
    \]
    A function is called \textit{concave} (concave down) if the inequality is flipped. \\[6pt]
    Additionally, convexity (concavity) can be determined by checking if $f''(x) \geq 0$ \newline ($f''(x) \leq 0$) holds for all $x \in [a,b]$. \\[6pt]
    Note that $f$ is convex if and only if $-f$ is concave. \\[6pt]
    Example (convex): $x^2, e^x$. Example (concave): $\ln{x}, \sqrt{x}$.
\end{tcolorbox}


\subsection{Cyclic Sum}
\begin{tcolorbox}[breakable]
    The \emph{cyclic sum} of a function $f$ over $n$ variables is the sum over \textit{all cyclic permutations} of its arguments:
    \[
    \sum_{\text{cyc}} f(a_1,a_2,\dots,a_n) = f(a_1,a_2,\dots,a_n) + f(a_2,a_3,\dots,a_1) + \dots +f(a_n,a_1,\dots,a_{n-1}).
    \]
    The number of terms is equal to the number of variables: $n$. \\[6pt]
    Example: $\textstyle f(a,b,c)=a^2b \implies \sum_{\text{cyc}}=a^2b+b^2c+c^2a$.
\end{tcolorbox}


\subsection{Symmetric Sum}
\begin{tcolorbox}[breakable]
    The \emph{symmetric sum} of a function $f$ over $n$ variables is the sum over \textit{all possible permutations} of its arguments:
    \[
    \sum_{\text{sym}} f(a_1,a_2,\dots,a_n) = \sum_{\sigma \in S_n} f(a_{\sigma(1)}, a_{\sigma(2)}, \dots,a_{\sigma(n)}),
    \]
    where $S_n$ is the set of all permutations of $\{1, 2, \dots, n\}$. \\[6pt]
    The number of terms equals to the number of all permutations of the variables $=n!$. \\[6pt]
    Example: $\textstyle f(a,b,c)=a^2b \implies \sum_{\text{sym}}=a^2b+a^2c+b^2a+b^2c+c^2a+c^2b$.
\end{tcolorbox}


\subsection{Elementary Symmetric Polynomials}
\begin{tcolorbox}[breakable]
    Let $t$ be a variable and $x_1, x_2, \dots, x_n$ be real numbers. Define:
    \begin{align*}
        P(x) &= \prod_{i=1}^n (t+x_i) = (t+x_1)(t+x_2)\dots(t+x_n) \\
        &= t^n + (x_1+\dots+x_n)t^{n-1} + (x_1x_2 + x_1x_3 + \dots)t^{n-2} + \dots \\
        & + (x_2x_3\dots x_n + x_1x_3\dots x_n + \dots)t + x_1x_2x_3\dots x_n \\
        &= 1 \cdot t^n + \left(\sum_{1\leq i\leq n} x_i\right) t^{n-1} + \left(\sum_{1\leq i < j\leq n} x_ix_j\right) t^{n-2} + \dots \\
        &+ \left(\sum_{1\leq i_1<\dots<i_{n-1}\leq n} x_{i_1}x_{i_2}\dots x_{i_{n-1}}\right) t + \prod_{i=1}^n x_i.
    \end{align*}
    In other words,
    \[
    P(x) = \prod_{i=1}^n (t+x_i) = c_0t^n + c_1t^{n-1} + c_2t^{n-2} + \dots + c_{n-1}t+c_n,
    \]
    where the coefficient $c_k$ is the $k$-th elementary symmetric sum:
    \begin{align*}
        &c_0 = 1, &&c_1 = \sum_{1\leq i \leq n} x_i, &&c_2 = \sum_{1\leq i < j \leq n} x_ix_j, \\
        &c_3 = \sum_{1\leq i < j < k \leq n} x_ix_jx_k, &&\dots, &&c_n = \prod_{i=1}^n x_i.
    \end{align*} \newpage
    In general, for $0 \leq k \leq n$
    \[
    c_k = \sum_{1 \leq i_1< i_2 < \dots < i_{k} \leq n} x_{i_1}x_{i_2}\dots x_{i_{k}}.
    \]
    Example: $\textstyle x_1=1, x_2=2, x_3=3 \implies (x+1)(x+2)(x+3) = x^3 + (1 + 2 + 3)x^2 + (1 \cdot 2 + 2 \cdot 3 + 3 \cdot 1) x + 1 \cdot 2 \cdot 3 = x^3 + 6x^2 + 11x + 6$.
\end{tcolorbox}


\subsection{Elementary Symmetric Mean}
\begin{tcolorbox}[breakable]
    Let $x_1, x_2, \dots, x_n$ be real numbers. The $k$-th elementary symmetric mean is defined as:
    \[
    d_k = \frac{c_k}{\binom{n}{k}} = \frac{1}{\binom{n}{k}} \sum_{1 \leq i_1< i_2 < \dots < i_{k} \leq n} x_{i_1}x_{i_2}\dots x_{i_{k}}.
    \]
    Example: $\textstyle x_1=1, x_2=2, x_3=3 \implies d_2 = \frac{c_2}{\binom{3}{2}}=\frac{11}{3}$.
\end{tcolorbox}


\section{Inequalities}


\subsection{AM-GM Inequality}
\begin{tcolorbox}[breakable]
    Let $a_1, a_2, \dots, a_n >0$. Then, the following inequality holds:
    \[
    \frac{a_1 + a_2 + \dots + a_n}{n} \geq \sqrt[n]{a_1 a_2 \dots a_n},
    \]
    with equality if and only if $a_1 = a_2 = \dots = a_n$.

    More precisely,
    \[
    \frac{1}{n} \sum_{i=1}^n a_i \geq \sqrt[n]{\prod_{i=1}^n a_i}.
    \]
    Example: $\textstyle \frac{a + b + c}{3} \geq \sqrt[3]{abc}$.
\end{tcolorbox}


\subsection{Weighted AM-GM Inequality}
\begin{tcolorbox}[breakable]
    Let $a_1, a_2, \dots, a_n > 0$ and $w_1, w_2, \dots, w_n$ be positive integers. Then, by AM-GM we have:
    \begin{align*}
        &\frac{\underbrace{a_1 + a_1 + \dots + a_1}_{w_1} + \underbrace{a_2 + a_2 + \dots + a_2}_{w_2} + \dots + \underbrace{a_n + a_n + \dots + a_n}_{w_n}}{w_1 + w_2 + \dots + w_n} \\
        &\geq \left( \underbrace{a_1 a_1 \dots a_1}_{w_1} \underbrace{a_2 a_2 \dots a_2}_{w_2} \dots \underbrace{a_n a_n \dots a_n}_{w_n} \right)^{\frac{1}{w_1 + w_2 + \dots + w_n}}.
    \end{align*}
    The above is equivalent to the following
    \begin{align*}
        \frac{w_1 a_1 + w_2 a_2 + \dots + w_n a_n}{w_1 + w_2 + \dots + w_n} \geq (a_1^{w_1} a_2^{w_2} \dots a_n^{w_n})^\frac{1}{w_1 + w_2 + \dots w_n}.
    \end{align*}
    More precisely,
    \[
    \frac{\sum_{i=1}^n w_i a_i}{\sum_{i=1}^n w_i} \geq \left( \prod_{i=1}^n a_i^{w_i} \right)^{\frac{1}{\sum_{i=1}^n w_i}}
    \]
    If we let $w_1, w_2, \dots, w_n \geq 0$ with $w_1 + w_2 + \dots + w_n = 1$, we have:
    \[
    w_1a_1 + w_2a_2 + \dots + w_na_n \geq a_1^{w_1} a_2^{w_2} \dots a_n^{w_n},
    \]
    or, more precisely,
    \[
    \sum_{i=1}^n w_ia_i \geq \prod_{i=1}^n a_i^{w_i}.
    \]
    Example: $\textstyle \frac{3a + 2b + c}{6} \geq \sqrt[6]{a^3 b^2 c}$.
\end{tcolorbox}


\subsection{Power Mean Inequality}
\begin{tcolorbox}[breakable]
    Let $a_1, a_2, \dots, a_n > 0$. Then, the $r$-th power mean is defined as:
    \[
    \mathcal{P}(r) =
    \left\{
    \begin{array}{l}
    \left( \frac{a_1^r + \dots + a_n^r}{n} \right)^{1/r} \quad \hfill r \neq 0,\vspace{10pt} \\
    \sqrt[n]{a_1 a_2 \dots a_n} \quad \hfill r = 0.
    \end{array}
    \right.
    \]
    Example:
    \begin{flalign*}
        &\bullet \quad r = -1: &\frac{n}{\frac{1}{a_1} + \frac{1}{a_2} + \dots + \frac{1}{a_n}} &= \frac{n}{\sum_{i=1}^n \frac{1}{a_i}} && \text{(Harmonic Mean)} \\
        &\bullet \quad r = 0: &\sqrt[n]{a_1 a_2 \dots a_n} &= \left( \prod_{i=1}^n a_i \right)^\frac{1}{n} && \text{(Geomteric Mean)} \\
        &\bullet \quad r = 1: &\frac{a_1 + a_2 + \dots + a_n}{n} &= \frac{1}{n} \sum_{i=1}^n a_i && \text{(Arithmetic Mean)} \\
        &\bullet \quad r = 2: &\sqrt{\frac{a_1^2 + a_2^2 + \dots + a_n^2}{n}} &= \sqrt{\frac{\sum_{i=1}^n a_i^2}{n}} && \text{(Quadratic Mean)}
    \end{flalign*}
    If $r > s$, then
    \[
    \mathcal{P}(r) \geq \mathcal{P}(s)
    \]
    with equality if and only if $a_1 = a_2 = \dots = a_n$. \\[6pt]
    Example: $\textstyle \mathcal{P}(2) \geq \mathcal{P}(1) \iff \sqrt{\frac{a^2 + b^2}{2}} \geq \frac{a+b}{2}$.
\end{tcolorbox}


\subsection{Weighted Power Mean Inequality}
\begin{tcolorbox}[breakable]
    Let $a_1, a_2, \dots a_n > 0$ and $w_1, w_2, \dots, w_n \geq 0$ with $w_1 + w_2 + \dots + w_n = 1$. Then, the $r$-th weighted power mean is defined as:
    \[
    \mathcal{P}(r) =
    \left\{
    \begin{array}{l}
    \left( w_1a_1^r + w_2a_2^r + \dots + w_na_n^r \right)^{1/r} \quad \hfill r \neq 0,\vspace{10pt} \\
    a_1^{w_1} a_2^{w_2} \dots a_n^{w_n} \quad \hfill r = 0.
    \end{array}
    \right.
    \]
    Similarly, if $r > s$, then
    \[
    \mathcal{P}(r) \geq \mathcal{P}(s)
    \]
    with equality if and only if $a_1 = a_2 = \dots = a_n$. \\[6pt]
    Example: $\textstyle (\frac{1}{6} a^3 + \frac{1}{3}b^3 + \frac{1}{2}c^3)^{1/3} \geq a^{1/6} b^{1/3} c^{1/2}$.
\end{tcolorbox}


\subsection{HM-GM-AM-QM Inequalities}
\begin{tcolorbox}[breakable]
    Let $a_1, a_2, \dots, a_n > 0$. Then:
    \begin{align*}
        0 < \text{HM} \leq \text{GM} &\leq \text{AM} \leq \text{QM} \\
        0 < \frac{n}{\frac{1}{a_1} + \frac{1}{a_2} + \dots + \frac{1}{a_n}} \leq \sqrt[n]{a_1 a_2 \dots a_n} &\leq \frac{a_1 + a_2 + \dots + a_n}{n} \leq \sqrt{\frac{a_1^2 + a_2^2 + \dots + a_n^2}{n}}.
    \end{align*}
    More precisely,
    \[
    0 < \frac{n}{\sum_{i=1}^n \frac{1}{a_i}} \leq \sqrt[n]{\prod_{i=1}^n a_i} \leq \frac{1}{n} \sum_{i=1}^n a_i \leq \sqrt{\frac{\sum_{i=1}^n a_i^2}{n}}.
    \]
    Example: $\textstyle \frac{2ab}{a+b} \leq \sqrt{ab} \leq \frac{a+b}{2} \leq \sqrt{\frac{a^2+b^2}{2}}$.
\end{tcolorbox}


\subsection{Bernoulli’s Inequality}
\begin{tcolorbox}[breakable]
    For all $x \geq -1$ and $r \geq 1$:
    \[
    (1 + x)^r \geq 1 + rx.
    \]
    Example: $\textstyle (1+x)^5 \geq 1+5x$.
\end{tcolorbox}


\subsection{Jensen's Inequality}
\begin{tcolorbox}[breakable]
    If $f$ is \emph{convex}, then:
    \[
    \frac{f(a_1) + f(a_2) + \dots + f(a_n)}{n} \geq f \left(\frac{a_1 + a_2 + \dots + a_n}{n} \right)
    \]
    with equality if and only if $f$ is \emph{linear} or $a_1 = a_2 = \dots = a_n$. \\[6pt]
    If we let $w_1, w_2, \dots, w_n \geq 0$ with $w_1 + w_2 + \dots + w_n = 1$, we have:
    \[
    w_1 f(a_1) + w_2 f(a_2) + \dots + w_n f(a_n) \geq f(w_1a_1 + w_2a_2 + \dots + w_na_n),
    \]
    or, more precisely,
    \[
    \sum_{i=1}^n w_i f(a_i) \geq f \left( \sum_{i=1}^n w_ia_i \right).
    \]
    The inequality is reversed if $f$ is concave. \\[6pt]
    Example: $\textstyle \sqrt{\frac{x+y}{2}} \geq \frac{\sqrt{x} + \sqrt{y}}{2}$.
\end{tcolorbox}


\subsection{Karamata's Inequality}
\begin{tcolorbox}[breakable]
    If $f$ is \emph{convex}, and $(a_i)$ \emph{majorizes} $(b_i)$, then:
    \[
    f(a_1) + f(a_2) + \dots f(a_n) \geq f(b_1) + f(b_2) + \dots f(b_n),
    \]
    or, more precisely,
    \[
    \sum_{i=1}^n f(a_i) \geq \sum_{i=1}^n f(b_i).
    \]
    The inequality is reversed if $f$ is concave. \\[6pt]
    Example: $\textstyle f(x) = x^2 \implies (4)^2 + (1)^2 \geq (2.5)^2 + (2.5)^2 \implies 17 \geq 12.5$.
\end{tcolorbox}


\subsection{Popoviciu's Inequality}
\begin{tcolorbox}[breakable]
    If $f$ is \emph{convex}, and $a, b, c > 0$, then:
    \begin{align*}
        &af(x) + bf(y) + cf(z) + (a+b+c) f\left(\frac{ax + by + cz}{a+b+c}\right) \geq \\[4pt]
        (&a+b) f\left(\frac{ax + by}{a+b}\right) + (b+c) f\left(\frac{by+cz}{b+c}\right) + (c+a) f\left(\frac{cz+ax}{c+a}\right)
    \end{align*}
    Particularly, if $a=b=c=1$, we have:
    \[
    \frac{f(x) + f(y) + f(c)}{3} + f \left( \frac{x+y+z}{3} \right) \geq \frac{2}{3} \left[ f\left(\frac{x+y}{2}\right) + f \left(\frac{y+z}{2}\right) + f \left(\frac{z+x}{2}\right) \right].
    \]
    Equality holds if and only if $f$ is \emph{linear} or $x=y=z$. \\[6pt]
    Example: $\textstyle f(x) = x^2 \implies \frac{(1)^2 + (2)^2 + (3)^2}{3} + \left(\frac{1+2+3}{3}\right)^2 \geq \frac{2}{3} \left[ \left(\frac{1+2}{2}\right)^2 + \left(\frac{2+3}{2}\right)^2 + \left(\frac{3+1}{2}\right)^2 \right] \implies \frac{26}{3} \geq \frac{25}{3}$.
\end{tcolorbox}


\subsection{Newton's Inequality}
\begin{tcolorbox}[breakable]
    For $x_1, x_2, \dots, x_n > 0$ and $k=1,2,\dots,n-1$, we have:
    \[
    d_i^2 \geq d_{i-1}d_{i+1},
    \]
    with equality if and only if $x_1 = x_2 = \dots = x_n$. \\[6pt]
    Example: $\textstyle x=1, y=2, z=3 \implies (\frac{xy + yz + zx}{3})^2 \geq (\frac{x+y+z}{3})\cdot xyz \\[6pt]
    \implies \left(\frac{1\cdot2 + 2\cdot3 + 3\cdot1}{3}\right)^2 \geq \frac{1+2+3}{3} (1\cdot2\cdot3) \implies \left(\frac{11}{3}\right)^2 \geq 2 \cdot 6 \implies 13.444 \geq 12$.
\end{tcolorbox}


\subsection{Maclaurin's Inequality}
\begin{tcolorbox}[breakable]
    For $x_1, x_2, \dots, x_n > 0$, we have:
    \[
    d_1 \geq \sqrt[2]{d_2} \geq \sqrt[3]{d_3} \geq \dots \geq \sqrt[n]{d_n}
    \]
    with equality if and only if $x_1 = x_2 = \dots = x_n$. \\[6pt]
    Equivalently, it can be written as:
    \[
    \frac{x_1+x_2+\dots+x_n}{n} \geq \sqrt{\frac{\sum_{1\leq i<j\leq n} x_ix_j}{\binom{n}{2}}} \geq \sqrt[3]{\frac{\sum_{1\leq i<j<k\leq n} x_ix_jx_k}{\binom{n}{3}}} \geq \dots \geq \sqrt[n]{x_1x_2\dots x_n}.
    \]
    Example: $\textstyle x=1, y=2, z=3 \implies \frac{x+y+z}{3} \geq \sqrt{\frac{xy + yz + zx}{3}} \geq \sqrt[3]{xyz} \\[6pt]
    \implies \frac{1+2+3}{3} \geq \sqrt{\frac{1\cdot2 + 2\cdot3 + 3\cdot1}{3}} \geq \sqrt[3]{1\cdot2\cdot3} \implies 2 \geq \frac{11}{3} \geq \sqrt[3]{6} \implies 2 \geq 1.915 \geq 1.817$.
\end{tcolorbox}


\subsection{Cauchy–Schwarz Inequality}
\begin{tcolorbox}[breakable]
    Let $a_1, a_2, \dots, a_n$, $b_1, b_2, \dots, b_n$ be real numbers. Then:
    \[
    (a_1^2 + a_2^2 + \dots + a_n^2)(b_1^2 + b_2^2 + \dots + b_n^2) \geq (a_1b_1 + a_2b_2 + \dots a_nb_n)^2,
    \]
    with equality if and only if there is a contant $\lambda \in \mathbb{R}$ such that $a_i = \lambda b_i$ for all $1 \leq i \leq n$. That is, if $\textstyle \frac{a_1}{b_1} = \frac{a_2}{b_2} = \dots = \frac{a_n}{b_n} = \lambda$. \\[6pt]
    More precisely,
    \[
    \left(\sum_{i=1}^n a_i^2\right)\left(\sum_{i=1}^n b_i^2\right) \geq \left(\sum_{i=1}^n a_ib_i\right)^2.
    \]
    Example: $\textstyle (a^2+b^2)(x^2+y^2) \geq (ax+by)^2 \implies (2^2 + 3^2)(4^2 + 5^2) \geq (2\cdot4+3\cdot5)^2 \\[6pt]
    \implies 13\cdot41\geq23^2 \implies 533\geq529$.
\end{tcolorbox}


\subsection{Titu's Lemma/Sedrakyan's Inequality/Engel's Form}
\begin{tcolorbox}[breakable]
    Let $a_1,a_2,\dots,a_n\geq0$ and $b_1,b_2,\dots,b_n>0$. Then:
    \[
    \frac{a_1^2}{b_1} + \frac{a_2^2}{b_2} + \dots + \frac{a_n^2}{b_n} \geq \frac{(a_1+a_2+\dots+a_n)^2}{b_1+b_2+\dots+b_n}.
    \]
    Example: $\textstyle \frac{a^2}{x} + \frac{b^2}{y} \geq \frac{(a+b)^2}{x+y}$.
\end{tcolorbox}


\subsection{Hölder's Inequality}
\begin{tcolorbox}[breakable]
    Let $a_1, a_2, \dots, a_n$, $b_1, b_2, \dots, b_n$, \dots, $z_1, z_2, \dots, z_n$ be positive real numbers, and let $\lambda_{a}, \lambda_{b}, \dots, \lambda_{z}$ be positive reals with $\lambda_{a} + \lambda_{b} + \dots + \lambda_{z} = 1$. Then:
    \[
    (a_1 + \dots + a_n)^{\lambda_{a}}(b_1 + \dots + b_n)^{\lambda_{b}}\dots(z_1 + \dots + z_n)^{\lambda_{z}} \geq a_1^{\lambda_{a}}b_1^{\lambda_{b}}\dots z_1^{\lambda_{z}} + \dots + a_n^{\lambda_{a}}b_n^{\lambda_{b}}\dots z_n^{\lambda_{z}},
    \]\newpage
    or, more precisely,
    \begin{align*}
        \underbrace{\left(\sum_{i=1}^n a_{i}\right)^{\lambda_a} \left(\sum_{i=1}^n b_{i}\right)^{\lambda_b}\dots \left(\sum_{i=1}^n z_{i}\right)^{\lambda_z}}_{m \text{ factors}} &\geq \sum_{i=1}^n \left(\underbrace{a_{i}^{\lambda_a}b_{i}^{\lambda_b}\dots z_{i}^{\lambda_z}}_{m \text{ variables}}\right) \\
        \prod_{j=1}^m \left(\sum_{i=1}^n a_{ij}\right)^{\lambda_j} &\geq \sum_{i=1}^n \left(\prod_{j=1}^m a_{ij}^{\lambda_j}\right).
    \end{align*}
    Example: $\textstyle m=3, n=2, \lambda_a=0.5, \lambda_b=0.3, \lambda_c=0.2, (a)=(1,3), (b)=(2,4), (c)=(5,6) \\[6pt]
    \implies (a_1+a_2)^{0.5}(b_1+b_2)^{0.3}(c_1+c_2)^{0.2} \geq a_1^{0.5}b_1^{0.3}c_1^{0.2}+a_2^{0.5}b_2^{0.3}c_2^{0.2} \\[6pt]
    \implies (1+3)^{0.5}(2+4)^{0.3}(5+6)^{0.2} \geq 1^{0.5}2^{0.3}5^{0.2}+3^{0.5}4^{0.3}6^{0.2} \implies 5.53 \geq 5.45$.
\end{tcolorbox}


\subsection{Minkowski Inequality}
\begin{tcolorbox}[breakable]
    Let $a_1, a_2, \dots, a_n$, $b_1, b_2, \dots, b_n$ be positive real numbers and $p>1$. Then:
    \[
    (a_1^p + a_2^p + \dots + a_n^p)^\frac{1}{p} + (b_1^p + b_2^p + \dots + b_n^p)^\frac{1}{p} \geq ((a_1 + b_1)^p + (a_2 + b_2)^p + \dots + (a_n + b_n)^p)^\frac{1}{p},
    \]
    or, more precisely,
    \[
    \left(\sum_{i=1}^n a_i^p\right)^\frac{1}{p} + \left(\sum_{i=1}^n b_i^p\right)^\frac{1}{p} \geq \left(\sum_{i=1}^n (a_i + b_i)^p\right)^\frac{1}{p}
    \]
    Example: $\textstyle p=2, a=(3,4), b=(6,8) \\[6pt]
    \implies \sqrt{a_1^2+a_2^2} + \sqrt{b_1^2+b_2^2} \geq \sqrt{(a_1+b_1)^2+(a_2+b_2)^2} \\[6pt]
    \implies  \sqrt{3^2+4^2} + \sqrt{6^2+8^2} \geq \sqrt{(3+6)^2+(4+8)^2} \\[6pt]
    \implies \sqrt{25} + \sqrt{100} \geq \sqrt{225} \implies 15 \geq 15$. (equal. why?)
\end{tcolorbox}


\subsection{Generalized Minkowski Inequality}
\begin{tcolorbox}[breakable]
    Let $a_1, a_2, \dots, a_n$, $b_1, b_2, \dots, b_n$, \dots, $z_1, z_2, \dots, z_n$ be positive real numbers, and $p>1$. Then:
    \[
    \underbrace{\left(\sum_{i=1}^n a_{i}^p\right)^\frac{1}{p} + \left(\sum_{i=1}^n b_{i}^p\right)^\frac{1}{p} + \dots + \left(\sum_{i=1}^n z_{i}^p\right)^\frac{1}{p}}_{m \text{ terms}} \geq \left(\sum_{i=1}^n\underbrace{(a_{i}+b_{i}+\dots+z_{i})^p}_{m \text{ terms}}\right)^\frac{1}{p}.
    \]
    More precisely,
    \[
    \sum_{j=1}^m \left(\sum_{i=1}^n a_{ij}^p\right)^\frac{1}{p} \geq \left[\sum_{i=1}^n \left(\sum_{j=1}^m a_{ij}\right)^p \right]^\frac{1}{p}
    \]
\end{tcolorbox}


\subsection{Young's Inequality}
\begin{tcolorbox}[breakable]
    Let $a,b\geq0$ and $p,q>1$ such that $\frac{1}{p}+\frac{1}{q}=1$. Then:
    \[
    \frac{a^p}{p} + \frac{b^q}{q} \geq ab
    \]
    with equality if and only if $a^p=b^q$. \\[6pt]
    Moreover, for increasing functions
    \[
    \int_0^a f(x)\,dx + \int_0^b f^{-1}(x)\,dx \geq ab
    \]
    with equality if and only if $f(a)=b$. \\[6pt]
    Example: $\textstyle a=2,b=3,p=3,q=\frac{3}{2} \implies \frac{2^3}{3}+\frac{3^{3/2}}{3/2} \geq 2\cdot3 \implies 6.13 \geq 6.$
\end{tcolorbox}


\subsection{Rearrangement Inequality}
\begin{tcolorbox}[breakable]
    Let $a_1 \leq a_2 \leq \dots \leq a_n$, $b_1 \leq b_2 \leq \dots \leq b_n$ be two sequences that are both increasing (or both decreasing). Then:
    \[
    a_1b_1 + a_2b_2 + \dots + a_nb_n \geq a_1b_{\sigma(1)} + a_1b_{\sigma(2)} + \dots + a_1b_{\sigma(n)} \geq a_1b_{n} + a_2b_{n-1} + \dots a_nb_{1}, 
    \]
    where $\sigma$ is a permutation function, which sends each of $1,2, \dots, n$ to a different value in $\{1,2,\dots,n\}$. \\[6pt]
    More precisely,
    \[
    \sum_{i=1}^n a_ib_i \geq \sum_{i=1}^n a_ib_{\sigma(i)} \geq \sum_{i=1}^n a_ib_{n+1-i}.
    \]
    In other words, the sum is \emph{maximized} when both sequences are ordered \emph{similarly} (both increasing or both decreasing), and is \emph{minimazied} when both sequences are ordered \emph{oppositely} (one increasing, the other decreasing). \\[6pt]
    Example: $\textstyle a^2+b^2+c^2 \geq ab + bc + ca; \\[6pt] 
    a=(1,3,5), \quad b=(2,4,6)
    \implies 1\cdot2+3\cdot4+5\cdot6 \geq 1\cdot4+3\cdot6+5\cdot2 \geq 1\cdot6+3\cdot4+5\cdot2 \implies 44 \geq 32 \geq 28$.
\end{tcolorbox}


\subsection{Chebyshev's Sum Inequality}
\begin{tcolorbox}[breakable]
    Let $a_1 \leq a_2 \leq \dots \leq a_n$, $b_1 \leq b_2 \leq \dots \leq b_n$ be two sequences that are both increasing (or both decreasing). Then:
    \[
    \frac{a_1b_1 + \dots + a_nb_n}{n} \geq \frac{a_1 + \dots + a_n}{n} \cdot \frac{b_1 + \dots + b_n}{n} \geq \frac{a_1b_n + \dots + a_nb_1}{n},
    \]
    or, more precisely,
    \[
    \frac{\sum_{i=1}^n a_ib_i}{n} \geq \frac{\sum_{i=1}^n a_i}{n} \times \frac{\sum_{i=1}^n b_i}{n} \geq \frac{\sum_{i=1}^n a_ib_{n+1-i}}{n}.
    \]
    Example: $a_1 \leq a_2 \leq a_3,\ b_1 \leq b_2 \leq b_3 \implies \textstyle a_1b_1 + a_2b_2 + a_3b_3 \geq \frac{1}{3} (a_1+a_2+a_3)(b_1+b_2+b_3)$.
\end{tcolorbox}


\subsection{Schur's Inequality}
\begin{tcolorbox}[breakable]
    Let $a,b,c \geq 0$ and $r>0$. Then:
    \[
    a^r(a-b)(a-c) + b^r(b-c)(b-a) + c^r(c-a)(c-b) \geq 0
    \]
    with equality if and only if $a=b=c$ or two of them are equal and the other is zero. \\[6pt]
    Example: $\textstyle r=1 \implies a^3+b^3+c^3+3abc\geq a^2(b+c) + b^2(c+a) + c^2(a+b)$.
\end{tcolorbox}


\subsection{Muirhead's Inequality}
\begin{tcolorbox}[breakable]
    Let $a_1,a_2,\dots,a_n\geq0$ and suppose that $(x_n)$ \emph{majorizes} $(y_n)$, $x\succ y$. Then:
    \[
    \sum_{\text{sym}} a_1^{x_1}a_2^{x_2}\dots a_n^{x_n} \geq \sum_{\text{sym}} a_1^{y_1}a_2^{y_2}\dots a_n^{y_n}.
    \]
    Example: $\textstyle (5,0) \succ (3,2) \implies x^5+y^5 \geq x^3y^2+x^2y^3$.
\end{tcolorbox}


\subsection{Nesbitt's Inequality}
\begin{tcolorbox}[breakable]
    Let $a,b,c>0$. Then:
    \[
    \frac{a}{b+c} + \frac{b}{c+a} + \frac{c}{a+b} \geq \frac{3}{2}
    \]
    with equality if and only if $a=b=c$. \\[6pt]
    Example: $\textstyle a=1,b=2,c=3 \implies \frac{1}{2+3} + \frac{2}{3+1} + \frac{3}{1+2} \geq \frac{3}{2} \implies \frac{1}{5} + \frac{2}{4} + \frac{3}{3} \geq \frac{3}{2} \implies1.7 \geq 1.5$.
\end{tcolorbox}


\subsection{Aczel's Inequality}
\begin{tcolorbox}[breakable]
    Let $a_1,\dots,a_n$, $b_1,\dots,b_n$ be positive real numbers. If $a_1^2\geq a_2^2+\dots+a_n^2$ and $b_1^2\geq b_2^2+\dots+b_n^2$, then:
    \[
    (a_1b_1 - a_2b_2 - \dots - a_nb_n)^2 \geq (a_1^2 - a_2^2 - \dots \ a_n^2)(b_1^2 - b_2^2 - \dots \ b_n^2)
    \]
    with equality if and only if the sequences are proportional. \\[6pt]
    More precisely,
    \[
    \left( a_1b_1 - \sum_{i=2}^n a_ib_i \right)^2 \geq \left( a_1^2 - \sum_{i=2}^n a_i^2 \right) \left( b_1^2 - \sum_{i=2}^n b_i^2 \right).
    \]
    Example: $(a)=(6,3,2), (b)=(5,4,1) \\[6pt]
    \implies (6\cdot5 - 3\cdot4 - 2\cdot 1)^2 \geq (6^2-3^2-2^2)(5^2-4^2-1^2) \implies 16^2 \geq 23\cdot8 \implies 256\geq184$.
\end{tcolorbox}


\subsection{Huygens Inequality}
\begin{tcolorbox}[breakable]
    Let $a_1,\dots,a_n$, $b_1,\dots,b_n$, $w_1,\dots,w_n$ be positive real numbers with $w_1+\dots+w_n=1$. Then:
    \[
    \prod_{i=1}^n (a_i+b_i)^{w_i} \geq \prod_{i=1}^n a_i^{w_i} + \prod_{i=1}^n b_i^{w_i}.
    \]
    Example: $\textstyle (a_i)=(6,11), (b_i)=(13,2), (w_i)=(3/4, 1/4) \\[6pt]
    \implies (6+13)^{3/4}(11+2)^{1/4} \geq 6^{3/4}11^{1/4} + 13^{3/4}2^{1/4} \implies 17.280\geq15.123$.
\end{tcolorbox}


\subsection{Heinz Mean Inequality}
\begin{tcolorbox}[breakable]
    Let $a,b>0$ and $0\leq\nu\leq1$. Then:
    \[
    \sqrt{ab} \leq \frac{a^{\nu}b^{1-\nu} + a^{1-\nu}b^{\nu}}{2} \leq \frac{a+b}{2}.
    \]
    Example: $\textstyle a=9,b=7,\nu=0.3 \\[6pt]
    \implies \sqrt{4\cdot17} \leq \frac{4^{0.3}17^{0.7} + 4^{0.7}17^{0.3}}{2} \leq \frac{4+17}{2} \implies 8.246 \leq 8.593 \leq 10.5$.
\end{tcolorbox}


\subsection{Mildorf's Inequality}
\begin{tcolorbox}[breakable]
    Let $k\geq-1$ be an integer and $a,b>0$. Then:
    \[
    \frac{(1+k)(a-b)^2+8ab}{4(a+b)} \geq \sqrt[k]{\frac{a^k+b^k}{2}}
    \]
    with equality if and only if $a=b$ or $k\in\{-1,1\}$, where the power mean $k=0$ is interpreted as the geometric mean $\sqrt{ab}$. Moreover, the inequality is flipped if $k<-1$. \\[6pt]
    Example: $\textstyle a=22, b=13,k=5 \\[6pt]
    \implies \frac{(1+5)(22-13)^2+8\cdot22\cdot13}{4(22+13)} \geq \sqrt[5]{\frac{22^5+13^5}{2}} \implies \frac{6\cdot9^2+2,288}{4\cdot35} \geq \sqrt[5]{\frac{5,153,632+371,293}{2}} \\[6pt]
    \implies \frac{2774}{140} \geq \sqrt[5]{2,762,462.5} \implies 19.814 \geq 19.420$.
\end{tcolorbox}


\section{Selected Inequalities}
\begin{equation}
    (a+b)(b+c)(c+a) \geq 8abc
\end{equation}
\vspace{13pt}
\begin{equation}
    \sqrt{1+\sqrt{a}} + \sqrt{1+\sqrt{a + \sqrt{a^2}}} + \ldots + \sqrt{1+\sqrt{a + \ldots +\sqrt{a^n}}} < na, \quad n\geq2, a\geq2, n\in\mathbb{N}
\end{equation}
\vspace{13pt}
\begin{equation}
    (n!)^2 \geq n^n, \quad n\in\mathbb{N}
\end{equation}
\vspace{13pt}
\begin{equation}
    \begin{aligned}
        \frac{1}{3} + \frac{2}{3\cdot5} + \frac{3}{3\cdot5\cdot7} + \ldots + \frac{n}{3\cdot5\dots(2n+1)} &< \frac{1}{2}, \quad n\in\mathbb{N} \\
        \sum_{k=1}^n \frac{k}{\prod_{j=1}^k (2j+1)} &< \frac{1}{2}, \quad n\in\mathbb{N}
    \end{aligned}
\end{equation}
\vspace{13pt}
\begin{equation}
    \begin{aligned}
        \frac{2^3+1}{2^3-1} \cdot \frac{3^3+1}{3^3-1} \cdot\ldots\cdot \frac{n^3+1}{n^3-1}&<\frac{3}{2}, \quad n\geq2,n\in\mathbb{N} \\
        \prod_{k=2}^n \frac{k^3+1}{k^3-1} &< \frac{3}{2}, \quad n\geq2,n\in\mathbb{N}
    \end{aligned}
\end{equation}
\vspace{13pt}
\begin{equation}
    \begin{aligned}
        1\cdot1! + 2\cdot2! + 3\cdot3! + \ldots + n\cdot n! < (n+1)!, \quad n\in\mathbb{N} \\
        \sum_{k=1}^n k\cdot k! < (n+1)!, \quad n\in\mathbb{N}
    \end{aligned}
\end{equation}
\vspace{13pt}
\begin{equation}
    \begin{aligned}
        \left(1+\frac{1}{2^2}\right)\left(1+\frac{1}{3^2}\right)\dots\left(1+\frac{1}{n^2}\right) &< 2, \quad n\geq2, n\in\mathbb{N} \\
        \prod_{k=2}^n \left(1+\frac{1}{k^2}\right) &< 2, \quad n\geq2, n\in\mathbb{N}
    \end{aligned}
\end{equation}
\vspace{13pt}
\begin{equation}
    \begin{aligned}
        x^8+y^8 \geq \frac{1}{128}, \quad x+y=1
    \end{aligned}
\end{equation}
\vspace{13pt}
\begin{equation}
    \begin{aligned}
        \left(x+\frac{1}{x}\right)^2 + \left(y+\frac{1}{y}\right)^2 \geq 12.5, \quad x,y>0,\,x+y=1
    \end{aligned}
\end{equation}
\vspace{13pt}
\begin{equation}
    \begin{aligned}
        \left(x_1+\frac{1}{x_1}\right)^2 + \left(x_2+\frac{1}{x_2}\right)^2 + \dots + \left(x_n+\frac{1}{x_n}\right)^2 &\geq n\left(n+\frac{1}{n}\right)^2, \quad x_k>0,\,\sum x_k=1 \\
        \sum_{k=1}^n \left(x_k+\frac{1}{x_k}\right)^2 &\geq n\left(n+\frac{1}{n}\right)^2, \quad x_k>0,\,\sum x_k=1
    \end{aligned}
\end{equation}
\vspace{13pt}
\begin{equation}
    \begin{aligned}
        n! \leq \left(\frac{n+1}{2}\right)^n, \quad n\in\mathbb{N}
    \end{aligned}
\end{equation}
\vspace{13pt}
\begin{equation}
    \begin{aligned}
        \frac{a_1}{a_2} + \frac{a_2}{a_3} + \dots + \frac{a_{n-1}}{a_n} + \frac{a_n}{a_1} &\geq n, \quad a_k>0 \\
        \sum_{k=1}^n \frac{a_k}{a_{k+1}} &\geq n, \quad a_k>0,\,a_{n+1}:=a_1
    \end{aligned}
\end{equation}
\vspace{13pt}
\begin{equation}
    \begin{aligned}
        \sqrt{a_1b_1} + \sqrt{a_2b_2} + \dots + \sqrt{a_nb_n} &\leq \sqrt{a_1+a_2+\dots+a_n}\cdot\sqrt{b_1+b_2+\dots+b_n}, \quad a_k,b_k>0 \\
        \sum_{k=1}^n \sqrt{a_k b_k} &\leq \sqrt{\sum_{k=1}^n a_k} \cdot \sqrt{\sum_{k=1}^n b_k}, \quad a_k,b_k>0
    \end{aligned}
\end{equation}
\vspace{13pt}
\begin{equation}
    \begin{aligned}
        \frac{1}{\sqrt{1}} + \frac{1}{\sqrt{2}} + \frac{1}{\sqrt{3}} + \dots + \frac{1}{\sqrt{n}} &\geq n\sqrt{\frac{2}{n+1}}, \quad n\in\mathbb{N} \\
        \sum_{k=1}^n \frac{1}{\sqrt{k}} &\geq n\sqrt{\frac{2}{n+1}}, \quad n\in\mathbb{N}
    \end{aligned}
\end{equation}
\vspace{13pt}
\begin{equation}
    \begin{aligned}
        \sqrt{a+\sqrt{a+\sqrt{a+\dots+\sqrt{a}}}} \leq \frac{1+\sqrt{1+4a}}{2}, \quad a\geq0
    \end{aligned}
\end{equation}
\vspace{13pt}
\begin{equation}
    \begin{aligned}
        \sqrt[n]{n} > \sqrt[n+1]{n+1}, \quad n\geq3
    \end{aligned}
\end{equation}
\vspace{13pt}
\begin{equation}
    \begin{aligned}
        1^1\cdot2^2\cdot3^3\cdot\ldots\cdot n^n &> (2n)!, \quad n\geq5,n\in\mathbb{N} \\
        \prod_{k=1}^n k^k &> (2n)!, \quad n\geq5,n\in\mathbb{N}
    \end{aligned}
\end{equation}
\vspace{13pt}
\begin{equation}
    \begin{aligned}
        \sqrt{1} + \sqrt{2} + \sqrt{3} + \dots + \sqrt{n} &> \frac{2}{3}n\sqrt{n}, \quad n\in\mathbb{N} \\
        \sum_{k=1}^n \sqrt{k} &> \frac{2}{3}n\sqrt{n}, \quad n\in\mathbb{N}
    \end{aligned}
\end{equation}
\vspace{13pt}
\begin{equation}
    \begin{aligned}
        e^x \geq x^e, \quad x\geq e
    \end{aligned}
\end{equation}
\vspace{13pt}
\begin{equation}
    \begin{aligned}
        \sqrt{2\sqrt[3]{3\sqrt[4]{4\sqrt[5]{5\dots\sqrt[n]{n}}}}} &< 2, \quad n\geq2,n\in\mathbb{N} \\
        2^{\frac{1}{2}} \cdot 3^{\frac{1}{2\cdot3}} \cdot 4^{\frac{1}{2\cdot3\cdot4}} \cdot \ldots \cdot n^{{\frac{1}{2\cdot3\cdot4\cdot\ldots\cdot n}}} = \prod_{k=2}^n k^{1/k!} &< 2, \quad n\geq2,n\in\mathbb{N}
    \end{aligned}
\end{equation}
\vspace{13pt}
\begin{equation}
    \begin{aligned}
        \sqrt{1+\sqrt{2+\sqrt{3+\sqrt{4+\dots+\sqrt{n}}}}} < 2, \quad n\in\mathbb{N}
    \end{aligned}
\end{equation}
\vspace{13pt}


\section{Selected Problems}
\begin{center}
    \textit{Soon.}
\end{center}


\section{Proofs}
\subsection{Proof of AM-GM Inequality using Induction}
\begin{tcolorbox}[breakable]
    \[
    \frac{a_1 + a_2 + \dots + a_n}{n} \geq \sqrt[n]{a_1 a_2 \dots a_n}
    \]
    \begin{enumerate}[label=\roman*.]
        \item Base case is true ($n=2$).
        \item $n$ is true $\implies n+1$ is true.
    \end{enumerate}
    \emph{Proof:} \\[6pt]
    Step 1:
    \[
    \frac{a_1 + a_2}{2} \geq \sqrt{a_1 a_2} \implies (\sqrt{a_1})^2 - 2\sqrt{a_1 a_2} + (\sqrt{a_2})^2 = (\sqrt{a_1} - \sqrt{a_2})^2 \geq 0.
    \]
    Step 2:
    \begin{align*}
        \frac{a_1 + \dots + a_n}{n} &\geq \sqrt[n]{a_1 \dots a_n} \implies\\
        \frac{a_1 + \dots + a_n + a_{n+1}}{n+1} &= \frac{ n \frac{a_1 + \dots + a_n}{n} + a_{n+1}}{n+1} \\  &\geq \left( \frac{a_1 + \dots + a_n}{n} \right)^{\frac{n}{n+1}} (a_{n+1})^{\frac{1}{n+1}} \\
        &\geq \left( \sqrt[n]{a_1 \dots a_n} \right)^{\frac{n}{n+1}} (a_{n+1})^{\frac{1}{n+1}} \\
        &= \sqrt[n+1]{a_1 \dots a_n a_{n+1}} 
    \end{align*}
    \hfill$\square$
\end{tcolorbox}


\subsection{Proof of AM-GM Inequality using Cauchy Induction}
\begin{tcolorbox}[breakable]
    \[
    \frac{a_1 + a_2 + \dots + a_n}{n} \geq \sqrt[n]{a_1 a_2 \dots a_n}
    \]
    \begin{enumerate}[label=\roman*.]
        \item Base case is true ($n=2$).
        \item $n$ is true $\implies 2n$ is true.
        \item $n$ is true $\implies n-1$ is true.
    \end{enumerate}
    \emph{Proof:} \\[6pt]
    Step 1:
    \[
    \frac{a_1 + a_2}{2} \geq \sqrt{a_1 a_2} \implies (\sqrt{a_1})^2 - 2\sqrt{a_1 a_2} + (\sqrt{a_2})^2 = (\sqrt{a_1} - \sqrt{a_2})^2 \geq 0.
    \]
    Step 2:
    \begin{align*}
        \frac{a_1 + \dots + a_n}{n} &\geq \sqrt[n]{a_1 \dots a_n} \implies\\
        \frac{a_1 + a_2 + \dots + a_{2n}}{2n} &= \frac{1}{2} \left( \frac{a_1 + a_2 + \dots + a_n}{n} + \frac{a_{n+1} + a_{n+2} + \dots + a_{2n}}{n} \right) \\  
        &\geq \frac{ \sqrt[n]{a_1 a_2 \dots a_n} + \sqrt[n]{a_{n+1} a_{n+2} \dots a_{2n}}}{2} \\
        &\geq \sqrt[2]{\sqrt[n]{a_1 a_2 \dots a_n} \cdot \sqrt[n]{a_{n+1} a_{n+2} \dots a_{2n}}} \\
        &= \sqrt[2n]{a_1 a_2 \dots a_{2n}} 
    \end{align*}
    Step 3:
    \begin{align*}
        \frac{a_1 + a_2 + \dots + a_{n-1} + a_n}{n} &\geq \sqrt[n]{a_1 a_2 \dots a_{n-1} a_n} \implies \\
        \frac{a_1 + a_2 + \dots + a_{n-1} +  \frac{a_1 + \dots + a_{n-1}}{n-1}}{n} &\geq \sqrt[n]{a_1 a_2 \dots a_{n-1} \cdot \frac{a_1 + \dots + a_{n-1}}{n-1}} \\
        \frac{(n-1)(a_1 + a_2 + \dots + a_{n-1}) + (a_1 + \dots + a_{n-1})}{n \cdot (n-1)} &= \frac{(n-1+1)(a_1 + a_2 + \dots + a_{n-1})}{n \cdot (n-1)} \\
        \frac{a_1 + a_2 + \dots + a_{n-1}}{n-1} &\geq \sqrt[n]{a_1 a_2 \dots a_{n-1} \cdot \frac{a_1 + \dots + a_{n-1}}{n-1}} \\
        \left( \frac{a_1 + a_2 + \dots + a_{n-1}}{n-1} \right)^n &\geq a_1 a_2 \dots a_{n-1} \cdot \frac{a_1 + \dots + a_{n-1}}{n-1} \\
        \left( \frac{a_1 + a_2 + \dots + a_{n-1}}{n-1} \right)^{n-1} &\geq a_1 a_2 \dots a_{n-1} \\
        \frac{a_1 + a_2 + \dots + a_{n-1}}{n-1} &\geq \sqrt[n-1]{a_1 a_2 \dots a_{n-1}}
    \end{align*}  
    \hfill$\square$
\end{tcolorbox}


\subsection{Proof of AM-GM Inequality using Jensen's Method}
\begin{tcolorbox}[breakable]
    Let $a_1, a_2, \dots, a_n > 0$ and $f(x) = \ln{x}$ be a \textit{concave} function on $(0, \infty)$. By Jensen's Inequality we have:
    \begin{align*}
        f \left(\frac{1}{n} \sum_{i=1}^n a_i \right) &\geq \frac{1}{n} \sum_{i=1}^n f(a_i) \\
        \ln{\left( \frac{a_1 + a_2 + \dots + a_n}{n} \right)} &\geq \frac{\ln{(a_1)} + \ln{(a_2)} + \dots + \ln{(a_n)}}{n} \\
        &= \frac{\ln{(a_1 a_2 \dots a_n)}}{n} \\
        &= \ln{(\sqrt[n]{a_1 a_2 \dots a_n})} \\
        e^{\ln{\left( \frac{a_1 + a_2 + \dots + a_n}{n} \right)}} &\geq e^{\ln{(\sqrt[n]{a_1 a_2 \dots a_n})}} \\
        \frac{a_1 + a_2 + \dots + a_n}{n} &\geq \sqrt[n]{a_1 a_2 \dots a_n}
    \end{align*}
    \hfill$\square$
\end{tcolorbox}

\end{document}
