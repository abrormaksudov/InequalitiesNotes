\documentclass[a4paper,11pt]{article}

\usepackage[top=1in, bottom=1in, left=1in, right=1in]{geometry}
\usepackage{hyperref}
\usepackage{amsmath, amssymb}
\usepackage{enumitem}
\usepackage{graphicx}
\usepackage[labelformat=empty]{caption}
\usepackage{tcolorbox}
\tcbuselibrary{breakable}
\usepackage{multicol}
\usepackage{tabularx}
\usepackage{ltablex}
\usepackage{centernot}

\title{Inequalities Notes}
\author{Abror Maksudov}
\date{Last Updated: \today}

\everymath{\displaystyle}

\begin{document}

\maketitle
\tableofcontents

\section{Introduction}


\section{Definitions}


\subsection{Majorization}
\begin{tcolorbox}
    Let $x = (x_1, x_2, \dots, x_n)$ and $y = (y_1, y_2, \dots, y_n)$ be non-increasing sequences of real numbers. Then $x$ is said to \emph{majorize} $y$, denoted $x \succ y$, if the following conditions are satisfied:
    \begin{enumerate}
        \item $x_1 \geq x_2 \geq \dots \geq x_n$ \emph{and} $y_1 \geq y_2 \geq \dots \geq y_n$\,;
        \item $\sum_{i=1}^n x_i = \sum_{i=1}^n y_i$\,;
        \item $\sum_{i=1}^k x_i \geq \sum_{i=1}^k y_i$ for all $k = 1, 2, \dots, n-1$.
    \end{enumerate}
    Example: $(3,1,0) \succ (2,1,1), \quad (12, 0,0) \succ (4,4,4)$.
\end{tcolorbox}


\subsection{Convex Function}
\begin{tcolorbox}
    A function $f : [a,b] \to \mathbb{R}$ is called \textit{convex} (concave up) on $[a,b]$ if and only if for all $x,y \in [a,b]$ and all $\lambda \in [0,1]$, the following inequality holds:
    \[
    \lambda f(x) + (1-\lambda)f(y) \geq f(\lambda x + (1-\lambda)y).
    \]
    A function is called \textit{concave} (concave down) if the inequality is flipped. \\[6pt]
    Additionally, convexity (concavity) can be determined by checking if $f''(x) \geq 0$ \newline ($f''(x) \leq 0$) holds for all $x \in [a,b]$. \\[6pt]
    Note that $f$ is convex if and only if $-f$ is concave. \\[6pt]
    Example (convex): $x^2, e^x$. Example (concave): $\ln{x}, \sqrt{x}$.
\end{tcolorbox}


\section{Inequalities}


\subsection{AM-GM Inequality}
\begin{tcolorbox}
    Let $a_1, a_2, \dots, a_n >0$. Then, the following inequality holds:
    \[
    \frac{a_1 + a_2 + \dots + a_n}{n} \geq \sqrt[n]{a_1 a_2 \dots a_n},
    \]
    with equality if and only if $a_1 = a_2 = \dots = a_n$.

    More precisely,
    \[
    \frac{1}{n} \sum_{i=1}^n a_i \geq \sqrt[n]{\prod_{i=1}^n a_i}.
    \]
    Example: $\textstyle \frac{a + b + c}{3} \geq \sqrt[3]{abc}$.
\end{tcolorbox}


\subsection{Weighted AM-GM Inequality}
\begin{tcolorbox}
    Let $a_1, a_2, \dots, a_n > 0$ and $w_1, w_2, \dots, w_n$ be positive integers. Then, by AM-GM we have:
    \begin{align*}
        &\frac{\underbrace{a_1 + a_1 + \dots + a_1}_{w_1} + \underbrace{a_2 + a_2 + \dots + a_2}_{w_2} + \dots + \underbrace{a_n + a_n + \dots + a_n}_{w_n}}{w_1 + w_2 + \dots + w_n} \\
        &\geq \left( \underbrace{a_1 a_1 \dots a_1}_{w_1} \underbrace{a_2 a_2 \dots a_2}_{w_2} \dots \underbrace{a_n a_n \dots a_n}_{w_n} \right)^{\frac{1}{w_1 + w_2 + \dots + w_n}}.
    \end{align*}
    The above is equivalent to the following
    \begin{align*}
        \frac{w_1 a_1 + w_2 a_2 + \dots + w_n a_n}{w_1 + w_2 + \dots + w_n} \geq (a_1^{w_1} a_2^{w_2} \dots a_n^{w_n})^\frac{1}{w_1 + w_2 + \dots w_n}.
    \end{align*}
    More precisely,
    \[
    \frac{\sum_{i=1}^n w_i a_i}{\sum_{i=1}^n w_i} \geq \left( \prod_{i=1}^n a_i^{w_i} \right)^{\frac{1}{\sum_{i=1}^n w_i}}
    \]
    If we let $w_1, w_2, \dots, w_n \geq 0$ with $w_1 + w_2 + \dots + w_n = 1$, we have:
    \[
    w_1a_1 + w_2a_2 + \dots + w_na_n \geq a_1^{w_1} a_2^{w_2} \dots a_n^{w_n},
    \]
    or, more precisely,
    \[
    \sum_{i=1}^n w_ia_i \geq \prod_{i=1}^n a_i^{w_i}.
    \]
    Example: $\textstyle \frac{3a + 2b + c}{6} \geq \sqrt[6]{a^3 b^2 c}$.
\end{tcolorbox}


\subsection{Power Mean Inequality}
\begin{tcolorbox}
    Let $a_1, a_2, \dots, a_n > 0$. Then, the $r$-th power mean is defined as:
    \[
    \mathcal{P}(r) =
    \left\{
    \begin{array}{l}
    \left( \frac{a_1^r + \dots + a_n^r}{n} \right)^{1/r} \quad \hfill r \neq 0,\vspace{10pt} \\
    \sqrt[n]{a_1 a_2 \dots a_n} \quad \hfill r = 0.
    \end{array}
    \right.
    \]
    Example:
    \begin{flalign*}
        &\bullet \quad r = -1: &\frac{n}{\frac{1}{a_1} + \frac{1}{a_2} + \dots + \frac{1}{a_n}} &= \frac{n}{\sum_{i=1}^n \frac{1}{a_i}} && \text{(Harmonic Mean)} \\
        &\bullet \quad r = 0: &\sqrt[n]{a_1 a_2 \dots a_n} &= \left( \prod_{i=1}^n a_i \right)^\frac{1}{n} && \text{(Geomteric Mean)} \\
        &\bullet \quad r = 1: &\frac{a_1 + a_2 + \dots + a_n}{n} &= \frac{1}{n} \sum_{i=1}^n a_i && \text{(Arithmetic Mean)} \\
        &\bullet \quad r = 2: &\sqrt{\frac{a_1^2 + a_2^2 + \dots + a_n^2}{n}} &= \sqrt{\frac{\sum_{i=1}^n a_i^2}{n}} && \text{(Quadratic Mean)}
    \end{flalign*}
    If $r > s$, then
    \[
    \mathcal{P}(r) \geq \mathcal{P}(s)
    \]
    with equality if and only if $a_1 = a_2 = \dots = a_n$. \\[6pt]
    Example: $\textstyle \mathcal{P}(2) \geq \mathcal{P}(1) \iff \sqrt{\frac{a^2 + b^2}{2}} \geq \frac{a+b}{2}$.
\end{tcolorbox}


\subsection{Weighted Power Mean Inequality}
\begin{tcolorbox}
    Let $a_1, a_2, \dots a_n > 0$ and $w_1, w_2, \dots, w_n \geq 0$ with $w_1 + w_2 + \dots + w_n = 1$. Then, the $r$-th weighted power mean is defined as:
    \[
    \mathcal{P}(r) =
    \left\{
    \begin{array}{l}
    \left( w_1a_1^r + w_2a_2^r + \dots + w_na_n^r \right)^{1/r} \quad \hfill r \neq 0,\vspace{10pt} \\
    a_1^{w_1} a_2^{w_2} \dots a_n^{w_n} \quad \hfill r = 0.
    \end{array}
    \right.
    \]
    Similarly, if $r > s$, then
    \[
    \mathcal{P}(r) \geq \mathcal{P}(s)
    \]
    with equality if and only if $a_1 = a_2 = \dots = a_n$. \\[6pt]
    Example: $\textstyle (\frac{1}{6} a^3 + \frac{1}{3}b^3 + \frac{1}{2}c^3)^{1/3} \geq a^{1/6} b^{1/3} c^{1/2}$.
\end{tcolorbox}


\subsection{HM-GM-AM-QM Inequalities}
\begin{tcolorbox}
    Let $a_1, a_2, \dots, a_n > 0$. Then:
    \begin{align*}
        0 < \text{HM} \leq \text{GM} &\leq \text{AM} \leq \text{QM} \\
        0 < \frac{n}{\frac{1}{a_1} + \frac{1}{a_2} + \dots + \frac{1}{a_n}} \leq \sqrt[n]{a_1 a_2 \dots a_n} &\leq \frac{a_1 + a_2 + \dots + a_n}{n} \leq \sqrt{\frac{a_1^2 + a_2^2 + \dots + a_n^2}{n}}.
    \end{align*}
    More precisely,
    \[
    0 < \frac{n}{\sum_{i=1}^n \frac{1}{a_i}} \leq \sqrt[n]{\prod_{i=1}^n a_i} \leq \frac{1}{n} \sum_{i=1}^n a_i \leq \sqrt{\frac{\sum_{i=1}^n a_i^2}{n}}.
    \]
    Example: $\textstyle \frac{2ab}{a+b} \leq \sqrt{ab} \leq \frac{a+b}{2} \leq \sqrt{\frac{a^2+b^2}{2}}$.
\end{tcolorbox}


\subsection{Bernoulli’s Inequality}
\begin{tcolorbox}
    For all $x \geq -1$ and $r \geq 1$:
    \[
    (1 + x)^r \geq 1 + rx.
    \]
    Example: $\textstyle (1+x)^5 \geq 1+5x$.
\end{tcolorbox}


\subsection{Jensen's Inequality}
\begin{tcolorbox}
    If $f$ is \emph{convex}, then:
    \[
    \frac{f(a_1) + f(a_2) + \dots + f(a_n)}{n} \geq f \left(\frac{a_1 + a_2 + \dots + a_n}{n} \right)
    \]
    with equality if and only if $f$ is \emph{linear} or $a_1 = a_2 = \dots = a_n$. \\[6pt]
    If we let $w_1, w_2, \dots, w_n \geq 0$ with $w_1 + w_2 + \dots + w_n = 1$, we have:
    \[
    w_1 f(a_1) + w_2 f(a_2) + \dots + w_n f(a_n) \geq f(w_1a_1 + w_2a_2 + \dots + w_na_n),
    \]
    or, more precisely,
    \[
    \sum_{i=1}^n w_i f(a_i) \geq f \left( \sum_{i=1}^n w_ia_i \right).
    \]
    The inequality is reversed if $f$ is concave. \\[6pt]
    Example: $\textstyle \sqrt{\frac{x+y}{2}} \geq \frac{\sqrt{x} + \sqrt{y}}{2}$.
\end{tcolorbox}


\subsection{Karamata's Inequality}
\begin{tcolorbox}
    If $f$ is \emph{convex}, and $(a_i)$ \emph{majorizes} $(b_i)$, then:
    \[
    f(a_1) + f(a_2) + \dots f(a_n) \geq f(b_1) + f(b_2) + \dots f(b_n),
    \]
    or, more precisely,
    \[
    \sum_{i=1}^n f(a_i) \geq \sum_{i=1}^n f(b_i).
    \]
    The inequality is reversed if $f$ is concave. \\[6pt]
    Example: $\textstyle f(x) = x^2 \implies (4)^2 + (1)^2 \geq (2.5)^2 + (2.5)^2 \implies 17 \geq 12.5$.
\end{tcolorbox}


\subsection{Popoviciu's Inequality}
\begin{tcolorbox}
    If $f$ is \emph{convex}, and $a, b, c > 0$, then:
    \begin{align*}
        &af(x) + bf(y) + cf(z) + (a+b+c) f\left(\frac{ax + by + cz}{a+b+c}\right) \geq \\[4pt]
        (&a+b) f\left(\frac{ax + by}{a+b}\right) + (b+c) f\left(\frac{by+cz}{b+c}\right) + (c+a) f\left(\frac{cz+ax}{c+a}\right)
    \end{align*}
    Particularly, if $a=b=c=1$, we have:
    \[
    \frac{f(x) + f(y) + f(c)}{3} + f \left( \frac{x+y+z}{3} \right) \geq \frac{2}{3} \left[ f\left(\frac{x+y}{2}\right) + f \left(\frac{y+z}{2}\right) + f \left(\frac{z+x}{2}\right) \right].
    \]
    Equality holds if and only if $f$ is \emph{linear} or $x=y=z$. \\[6pt]
    Example: $\textstyle f(x) = x^2 \implies \frac{(1)^2 + (2)^2 + (3)^2}{3} + \left(\frac{1+2+3}{3}\right)^2 \geq \frac{2}{3} \left[ \left(\frac{1+2}{2}\right)^2 + \left(\frac{2+3}{2}\right)^2 + \left(\frac{3+1}{2}\right)^2 \right] \implies \frac{26}{3} \geq \frac{25}{3}$.
\end{tcolorbox}



\subsection{Young's Inequality}
\subsection{Cauchy–Schwarz Inequality}
\subsection{Muirhead's Inequality}
\subsection{Hölder's Inequality}
\subsection{Minkowski Inequality}
\subsection{Generalized Minkowski Inequality}
\subsection{Chebyshev's Sum Inequality}
\subsection{Rearrangement Inequality}
\subsection{Schur's Inequality}
\subsection{Generalized Schur's Inequality}
\subsection{Newton's Inequality}
\subsection{Maclaurin's Inequality}
\subsection{Aczel's Inequality}
\subsection{Huygens Inequality}
\subsection{Heinz Inequality}
\subsection{Nesbitt's Inequality}
\subsection{Cesàro's Inequality}
\subsection{Mildorf's Inequality}


\section{Selected Inequalities}
\section{Proofs}
\subsection{Proof of AM-GM Inequality using Induction}
\begin{tcolorbox}[breakable]
    \[
    \frac{a_1 + a_2 + \dots + a_n}{n} \geq \sqrt[n]{a_1 a_2 \dots a_n}
    \]
    \begin{enumerate}[label=\roman*.]
        \item Base case is true ($n=2$).
        \item $n$ is true $\implies n+1$ is true.
    \end{enumerate}
    \emph{Proof:} \\[6pt]
    Step 1:
    \[
    \frac{a_1 + a_2}{2} \geq \sqrt{a_1 a_2} \implies (\sqrt{a_1})^2 - 2\sqrt{a_1 a_2} + (\sqrt{a_2})^2 = (\sqrt{a_1} - \sqrt{a_2})^2 \geq 0.
    \]
    Step 2:
    \begin{align*}
        \frac{a_1 + \dots + a_n}{n} &\geq \sqrt[n]{a_1 \dots a_n} \implies\\
        \frac{a_1 + \dots + a_n + a_{n+1}}{n+1} &= \frac{ n \frac{a_1 + \dots + a_n}{n} + a_{n+1}}{n+1} \\  &\geq \left( \frac{a_1 + \dots + a_n}{n} \right)^{\frac{n}{n+1}} (a_{n+1})^{\frac{1}{n+1}} \\
        &\geq \left( \sqrt[n]{a_1 \dots a_n} \right)^{\frac{n}{n+1}} (a_{n+1})^{\frac{1}{n+1}} \\
        &= \sqrt[n+1]{a_1 \dots a_n a_{n+1}} 
    \end{align*}
    \hfill$\square$
\end{tcolorbox}


\subsection{Proof of AM-GM Inequality using Forward-Backward Induction (a.k.a. Cauchy Induction)}
\begin{tcolorbox}[breakable]
    \[
    \frac{a_1 + a_2 + \dots + a_n}{n} \geq \sqrt[n]{a_1 a_2 \dots a_n}
    \]
    \begin{enumerate}[label=\roman*.]
        \item Base case is true ($n=2$).
        \item $n$ is true $\implies 2n$ is true.
        \item $n$ is true $\implies n-1$ is true.
    \end{enumerate}
    \emph{Proof:} \\[6pt]
    Step 1:
    \[
    \frac{a_1 + a_2}{2} \geq \sqrt{a_1 a_2} \implies (\sqrt{a_1})^2 - 2\sqrt{a_1 a_2} + (\sqrt{a_2})^2 = (\sqrt{a_1} - \sqrt{a_2})^2 \geq 0.
    \]
    Step 2:
    \begin{align*}
        \frac{a_1 + \dots + a_n}{n} &\geq \sqrt[n]{a_1 \dots a_n} \implies\\
        \frac{a_1 + a_2 + \dots + a_{2n}}{2n} &= \frac{1}{2} \left( \frac{a_1 + a_2 + \dots + a_n}{n} + \frac{a_{n+1} + a_{n+2} + \dots + a_{2n}}{n} \right) \\  
        &\geq \frac{ \sqrt[n]{a_1 a_2 \dots a_n} + \sqrt[n]{a_{n+1} a_{n+2} \dots a_{2n}}}{2} \\
        &\geq \sqrt[2]{\sqrt[n]{a_1 a_2 \dots a_n} \cdot \sqrt[n]{a_{n+1} a_{n+2} \dots a_{2n}}} \\
        &= \sqrt[2n]{a_1 a_2 \dots a_{2n}} 
    \end{align*}
    Step 3:
    \begin{align*}
        \frac{a_1 + a_2 + \dots + a_{n-1} + a_n}{n} &\geq \sqrt[n]{a_1 a_2 \dots a_{n-1} a_n} \implies \\
        \frac{a_1 + a_2 + \dots + a_{n-1} +  \frac{a_1 + \dots + a_{n-1}}{n-1}}{n} &\geq \sqrt[n]{a_1 a_2 \dots a_{n-1} \cdot \frac{a_1 + \dots + a_{n-1}}{n-1}} \\
        \frac{(n-1)(a_1 + a_2 + \dots + a_{n-1}) + (a_1 + \dots + a_{n-1})}{n \cdot (n-1)} &= \frac{(n-1+1)(a_1 + a_2 + \dots + a_{n-1})}{n \cdot (n-1)} \\
        \frac{a_1 + a_2 + \dots + a_{n-1}}{n-1} &\geq \sqrt[n]{a_1 a_2 \dots a_{n-1} \cdot \frac{a_1 + \dots + a_{n-1}}{n-1}} \\
        \left( \frac{a_1 + a_2 + \dots + a_{n-1}}{n-1} \right)^n &\geq a_1 a_2 \dots a_{n-1} \cdot \frac{a_1 + \dots + a_{n-1}}{n-1} \\
        \left( \frac{a_1 + a_2 + \dots + a_{n-1}}{n-1} \right)^{n-1} &\geq a_1 a_2 \dots a_{n-1} \\
        \frac{a_1 + a_2 + \dots + a_{n-1}}{n-1} &\geq \sqrt[n-1]{a_1 a_2 \dots a_{n-1}}
    \end{align*}  
    \hfill$\square$
\end{tcolorbox}


\subsection{Proof of AM-GM Inequality using Jensen's Method}
\begin{tcolorbox}
    Let $a_1, a_2, \dots, a_n > 0$ and $f(x) = \ln{x}$ be a \textit{concave} function on $(0, \infty)$. By Jensen's Inequality we have:
    \begin{align*}
        f \left(\frac{1}{n} \sum_{i=1}^n a_i \right) &\geq \frac{1}{n} \sum_{i=1}^n f(a_i) \\
        \ln{\left( \frac{a_1 + a_2 + \dots + a_n}{n} \right)} &\geq \frac{\ln{(a_1)} + \ln{(a_2)} + \dots + \ln{(a_n)}}{n} \\
        &= \frac{\ln{(a_1 a_2 \dots a_n)}}{n} \\
        &= \ln{(\sqrt[n]{a_1 a_2 \dots a_n})} \\
        e^{\ln{\left( \frac{a_1 + a_2 + \dots + a_n}{n} \right)}} &\geq e^{\ln{(\sqrt[n]{a_1 a_2 \dots a_n})}} \\
        \frac{a_1 + a_2 + \dots + a_n}{n} &\geq \sqrt[n]{a_1 a_2 \dots a_n}
    \end{align*}
    \hfill$\square$
\end{tcolorbox}


\section{Selected Problems}
\begin{enumerate}
    \item $-$
\end{enumerate}





\end{document}
